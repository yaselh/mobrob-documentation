Der Softwareteil besteht grundsätzlich aus vier Komponenten. Die Lokalisierungskomponente ermöglicht die Detektion der Tasse und des Turtlebots. Sie stellt zudem diverse Werkzeuge zur automatischen Kamerakalibrierung, Poseschätzung der Tasse, Manipulation der Punktwolken und  Annotation der Datensätze zur Verfügung. Die Bahnplanungskomponente führt die Bahnplanung des Roboterarms zwischen der erkannten Tasse und dem erkannten Turtlebot aus. Die Greifkomponente steuert den Fingerabstand, um die erkannte Tasse zu greifen bzw. die Tasse auf den erkannten Turtlebot zu legen. Letztendlich dient die High-Level-Steuerung zur Koordination der verschiedenen Softwarekomponenten und bietet die Schnittstelle zur Kommunikation mit dem zentralen State Manager der drei Gruppen an.\\
