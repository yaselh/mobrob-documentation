Die Hardware besteht aus dem Greifer Schunk PG70 mit 3D-gedruckten Fingern, die durch ein Muster aus Haushaltsgummis erweitert wurden. Die Kommunikation mit dem Greifer findet über eine CAN Schnittstelle statt, die per USB am Shuttle Computer angeschlossen ist. Ein externes Netzteil versorgt den Greifer mit Strom.

Der Greifer wird durch den \glqq schunk\textunderscore canopen\textunderscore driver\grqq des FZI angesprochen, welches eine Action zur Konfiguration der Backenabstände anbietet. Das Paket wird von \glqq cup\textunderscore gripper\grqq verwendet, ein ROS Paket mit den beiden Actionen GrabCup und ReleaseCup. GrabCup akzeptiert ein cup\textunderscore diameter als Parameter, um den Abstand der Finger auf die jeweilige Tasse anzupassen. Innerhalb des Packets wird der Greifer durch \glqq gripper \grqq abstrahiert, was einen vereinfachten Tausch der Hardware ermöglichen würde.
