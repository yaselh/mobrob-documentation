Als Ausgangspunkt für die Realisierung der Aufgabe wurde uns sowohl diverse Hardware als auch Software zu Verfügung gestellt. So stand uns für das Greifen der Tasse ein UR5 Roboter von Universal Robots, welcher bereits auf einem Tisch befestigt wurde. Zur Lokalisierung der Tasse waren eine Kinect bzw. alternativ eine Intel Realsense Kamera verfügbar. Außerdem standen uns für die Entwicklung der Software sowohl vorkonfigurierte Rechner in Poolräumen, als auch ein an den UR5 Roboter angeschlossener Shuttle PC zur Verfügung. Als Grundlage für die zu entwickelnde Software dienten ROS Indigo bzw. Kinetic.