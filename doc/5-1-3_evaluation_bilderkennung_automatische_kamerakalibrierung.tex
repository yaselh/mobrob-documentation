Generell funktioniert der beschriebene Ansatz recht robust, jedoch kommt es gelegentlich zu Fehlerkennungen bei der Detektion der Kanten des Tisches, welche zur Berechnung einer falschen Transformation vom Kamera- ins Weltkoordinaten System führen. Hier könnten weitere Überprüfungen auf die Plausibilität des Detektionsergebnisses Abhilfe schaffen.
Ein weiterer Nachteil des Ansatzes besteht darin, dass er auf den Labortisch zugeschnitten wurde. Für den Einsatz in einer anderen umgebung müsste er entsprechend angepasst werden. Ggf. wäre Alternativ eine Kalibrierung mittels eines entsprechenden Kalibrierungsobjekts bzw. Musters denkbar um die Kalibrierung flexibler zu gestalten.