In diesem Abschnitt sollen die für unsere Aufgabenstellung relevanten Themen aus der aktuellen Forschung vorgestellt werden.

Die Lokalisierung der Tasse lässt sich auf vielfältige Art und Weise realisieren. Eine Strategie, die wir auch in unserem Ansatz verfolgen, ist es die Tasse in den Bilddaten zu detektieren und die exakten Abmessungen anschließend aus der Punktwolke zu berechnen. Aus diesem Grund ist für uns vor allem das große Feld der Objekterkennung interessant. Hier zeichnet sich in den letzten Jahren deutlich der Trend hin zu Verfahren des maschinellen Lernens, insbesondere des Deep Learnings, ab. Vorangetrieben wird dies insbesondere durch die Entwicklung leistungsfähiger Hardware, insbesondere von GPUs, sowie der Verfügbarkeit von großen Trainingsdatensätzen. Deep-Learning basierte Ansätze wurden in folgenden für uns relevanten Bereichen bereits eingesetzt und übertrafen bisherige Ansätze deutlich.

\paragraph{Bild Klassifikation}
Ein für uns relevanter Bereich ist die Bild Klassifikation. Das Ziel hierbei ist es in Bildern präsente Objekte korrekt zu klassifizieren. Aktuelle Ansätze wie setzen hierbei auf der Verwendung von Convolutional Neural Networks (CNNs). Wichtige Publikationen sind u.A. \cite{He.2016} und \cite{Szegedy.20151211}

\paragraph{Objekt Detektion}
Das Ziel bei der Objekt Detektion ist es, Objekte einer bestimmten Klasse in einem Bild zu detektieren. Für die Klassifikation von Bildern entworfene Neuronale Netze werden häufig auch für die Objekt Detektion eingesetzt, indem sie mit anderen Verfahren kombiniert werden. Für konkrete Beispiele zur Objekt Detektion mittels Deep Learning siehe auch \cite{Dai.20160621} und \cite{Ren.2017}. 

\paragraph{Semantische Segmentierung}
Insbesondere im Bereich des autonomen Fahrens gewannen Ansätze zur Semantischen Segmentierung in den letzten Jahren stark an Popularität. Das Ziel einer semantischen Segmentierung ist es, jedem Pixel eines Eingabebildes eine Objektklasse zuzuordnen. Realisiert wird dies zumeist über sogenannte Fully Convolutional Neural Networks. Wichtige Publikationen im Bereich der semantischen Segmentierung sind u.A. \cite{Shelhamer.2017}, \cite{Chen.2016} und \cite{Badrinarayanan.2017}.

Neben den vorgestellten Deep Learning Ansätzen kann es auch durchaus angebrachter sein andere Verfahren des Maschinellen Lernens zu verwenden. Gründe hierfür könnten unter anderem die Knappheit von Trainingsdaten, Limitierte Hardware Resourcen als auch aufgabenspezifische Eigenheiten sein. Um insbesondere im Bezug auf limitierte Hardware Resourcen und die Knappheit von Trainingsdaten Abhilfe zu schaffen kann es insbesondere bei überschaubaren Problemstellungen gerechtfertigt sein auch flache Klassifikatoren einzusetzen. Ein populärer Vertreter der flachen Klassifikationsverfahren sind die Support Vector Machines. Diese wurden in \cite{Dalal.2005} erstmals in Kombination mit HOG Features für die Personen Detektion in Bilddaten eingesetzt.

Eine Alternative wäre es gewesen, die Tasse nicht mithilfe der Bilddaten, sondern mithilfe der Punktwolke zu detektieren. Auch hier existieren Deep Learning basierte Ansätze wie z.B. \cite{Maturana.2015} oder \cite{Engelcke.20170305}.