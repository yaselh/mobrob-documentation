Die Aufgabe der Bahnplanung war es, den PG70 unter Verwendung des UR5 zur Position der Kaffeetasse zu manövrieren, um diese zu greifen und anschließend sicher und ohne das Verschütten jeglicher Flüssigkeit auf den Turtlebot abzustellen. Dabei gab es mehrere Problem, die es zu bewältigen gab.
\newline
Zum einen musste dem Roboterarm gesagt werden, wo sich die Tasse bzw. der Turtlebot befindet und wie er die beiden Ziele ansteuern soll. Dabei musste sowohl eine Selbstkollision, als auch eine Kollision mit der Umgebung vermieden werden. Des Weiteren war es auf der Strecke von der Kaffeetasse zum Turtlebot notwendig, dass die Tasse ihre aufrechte Orientierung beibehält, damit der Kaffee bei dem Transport nicht verschüttet wird.
% Für die Lokalisierung war es außerdem hilfreich, dass der UR5 sich, wenn er nicht gerade die Tasse befördert, in einer Home-Konfiguration befindet, in der er nicht unnötigerweise in den Bildbereich des Tischen hineinragt und somit die Lokalisierung erschwert oder die Tasse sogar völlig verdeckt.
\newline
Um diese Aufgabe zu erfüllen und die dabei auftretenden Probleme zu lösen, hatten wir uns zwei Optionen herausgesucht, zwischen denen wir uns entscheiden mussten:
\begin{enumerate}
	\item Die Bahnplanung mit dem Motion Planning Framework MoveIt \cite{MoveIt} realisiern.
	\item Die Bahnplanung mit der FZI-hauseigenen Motion Pipeline \cite{FZIPipeline} realisiern.
\end{enumerate}
MoveIt ist eine Stage of the Art Software zur Steuerung von Robotern. Sie beinhaltet verschiedene Lösungen der inversen Kinematik (IK), unterschiedliche Planer zur Trajektorienerstellung und Optimierungsmöglichkeiten der gefundenen Trajektorien. MoveIt ist mit all seinen Möglichkeiten und Anwendungsgebieten sehr umfangreich und komplex. Dafür existieren jedoch ausführliche Dokumentationen und Tutorials, sowie eine aktive Community, die sich bei Fragen und Probleme untereinander zu Rat steht.
\newline
Die FZI-eigene Motion Pipeline ist ein Framework zur einfachen Ausführung von Trajektorien auf bekannten Kinematiken. Dank einer rqt-GUI ist es besonders benutzerfreundlich, zum Beispiel Bahnen für einen Roboterarm zu lernen. Diese Trajektorien können dann gespeichert werden und müssen nur noch vor der Ausführung geladen werden. Auch die FZI Motion Pipeline besitzt eine Dokumentation, welche im Vergleich zu MoveIt allerdings weniger umfangreich ist.
\newline
Letzten Endes haben wir uns für den Einsatz von MoveIt entschieden. Denn obwohl dieses Framework für unsere Anwendungszwecke übermächtig ist und aufgrund der Komplexität eine länger Einarbeitsphase benötigt, wollten wir uns trotzdem im Bereich der Bahnplanung keine Grenzen setzen. MoveIt schien uns daher als das Mittel der Wahl.
\newline
Der erste Schritt bei der Integrierung von MoveIt in unser Projekt war es daher, das Framework für unseren genauen Roboteraufbau zu konfigurieren. Unter Verwendung des MoveIt Setup Assistant \cite{MoveItAssistant} erzeugten wir uns, mit Hilfe unseres selbst erstellten URDF-Modells, unser eigenes ROS-Paket mit allen Konfigurations- und Launch-Dateien, die wir für die Benutzung von MoveIt auf unserer Hardware benötigten.
%Da wir die Bahnplanung in Python implementierten mussten wir außerdem noch das ROS-Paket "`Moveit_Commander"' \cite{MoveItCommander} zu unserem Projekt hinzufügen.
\newline
Zur Bahnplanung in MoveIt verwendeten wir Planer aus der Open Motion Planning Library (OMPL) verschiedene Planner, wobei wir uns Am Ende für den RRTConnectDefault entschieden. Dieser implementiert den RRT-Connect Algorithmus und lieferte uns die zuverlässigsten Ergebnisse. Bei den IK-Implementierungen gab es keine beobachtbaren Unterschiede.
\newline
Für die eigentliche Bestimmung der Trajektorie des UR5 stellt MoveIt zwei Methoden zur Verfügung.
\begin{enumerate}
	\item Bahnplanung vom momentanen Zustand zur Zielpose.
	\item Bahnplanung vom momentanen Zustand über eine Liste von Zielposen.
\end{enumerate}
Bei der ersten Variante wird nur der momentane Zustand des Roboters gesetzt. Anschließend wird mit der Methode \textit{set_target_pose} die Zielpose gesetzt und mit \textit{plan} die Bewegung geplant. Die zweite Methodik basiert auf der Bahnplanung über mehrere Wegpunkte. Die Methode \textit{compute_cartesian_path} bestimmt dabei aus mehreren Posen eine Trajektorie. Wir entschieden uns schließlich für die zweite Variante, da diese ohne Pfad-Constraints und weitere Optimierungsschritte für unsere Aufgabe optimale Bahnen bestimmte und auch die Ausrichtung der Tasse beibehielt.
