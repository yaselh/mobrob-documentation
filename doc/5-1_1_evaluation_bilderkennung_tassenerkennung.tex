Während die Tassenerkennung insgesamt recht robust funktioniert (Hier ggf. noch quantitative oder qualitative Analyse) gibt es durchaus noch Potential zur Verbesserung.

So könnte der für die Klassifikation der Tasse verwendete SVM Klassifikator mit variableren Trainingsdaten trainiert werden um die Generalisierung des selbigen zu verbessern und die Erkennung auch unter schwierigen bzw. varriierenden Lichtverhältnisse und mit verschiedenen Tassen robuster zu machen. Außerdem könnte der Sliding Window Ansatz, der im Moment als Standardverfahren für die Gewinnung der zu klassifizierenden Bildsegmente aktiv ist mit einem Interest Point Detektor ersetzt bzw. kombiniert werden um die Anzahl der Segmente zu reduzieren und die Erkennung damit zu beschleunigen.

Weiterer Spielraum zur Verbesserung besteht bei der Erkennung der Orientierung der Tasse. Wie beschrieben versucht der Algorithmus die Orientierung des Henkels anhand von dessen Position in der Punktwolke zu erkennen. Es stellte sich jedoch bei Tests heraus, dass der Henkel, insbesondere wenn sich die Tasse an Randpositionen des Tisches befindet, in der Punktwolke kaum auszumachen ist, weshalb die Erkennung hier häufiger versagt. Ggf. könnte man die Orientierung stattdessen mit Hilfe der Bilddaten z.B. über das Training eines Neuronalen Netzes zu realisieren.