Im Bereich der Bahnplanung genie�en auf Abtastung basierende Algorithmen wie \textit{probabilistic roadmap} (PRM) oder \textit{rapidly-exploring random tree} (RRT) gro�e Beliebtheit. Die liegt nicht zuletzt an ihrer asymptotischen Optimalit�t \cite{SamplingAlgos}.
\newline
Die PRM-Planungsmethode \cite{PRM} besteht aus zwei Phasen:
\begin{enumerate}
	\item Learning-Phase
	\item Query-Phase
\end{enumerate}
In der Learning-Phase wird ein kollisionsfreier Graph im Konfigurationsraums des Roboters erstellt, indem zuf�llige Punkte im Raum auf Kollision untersucht wird und anschlie�end mit Hilfe eines lokalen Pfadplaners mit dem bestehenden Graph, ebenfalls kollisionsfrei, verbunden wird. In der Query-Phase wird durch einen Graphsuchalgorithmus ein Pfad vom Anfangs- zum Zielpunkt bestimmt.
\newline
Die RRT-Methode \cite{RRT} ben�tigt keine Learning-Phase. Zwar werden auch hier zuf�llig Punkte im Konfigurationsraum ausgew�hlt und auf Kollision mit der Umgebung gepr�ft, der Aufbau des Graphes, welcher die Form eines Suchbaums besitzt, geschieht jedoch f�r jede Pfadanfrage von Neuem. Somit ben�tigt eine Suchanfrage bei der RRT-Methode potentiell l�nger im Vergleich zu PRM, der Graph ber�cksichtigt jedoch auch dynamische Objekte, deren Position sich in der Umgebung ver�ndern kann.
\newline
RRT-Connect \cite{RRTConnect} ist eine Erweiterung von der RRT-Planungsmethode. Hier werden zwei Suchb�ume gleichzeitig aufgebaut. Einer beginnt vom Startzustand, der andere vom Zielzustand. Nach einer vorgegebenen Anzahl von Iterationen wird dann versucht, die beiden B�ume direkt zu verbinden. Dabei wird eine greedy-Heuristik verwendet, um die Wahrscheinlichkeit zu erh�hen, dass sich die beiden B�ume aufeinander zubewegen.