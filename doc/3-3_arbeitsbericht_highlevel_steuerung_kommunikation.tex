Zur Koordination der verschiedenen Softwarekomponenten der Gruppe wird ein Managementknoten benötigt, der gleichermaßen die Kommunikation mit anderen Gruppen leitet. Schnittstellen zwischen den Gruppen wurden in mehreren gemeinsamen Treffen einstimmig beschlossen und selbstständig vom jeweiligen Team implementiert. Für eine vereinfachte Konversation erhielt die vorgestellte Komponente den Namen "Cup Acceptor", da die Tasse in Empfang genommen wird. Die Entwicklung findet zuerst unter dem Namen "Highlevelcontrol" statt, jedoch wird das Paket in "cup\textunderscore acceptor\textunderscore manager" umbenannt. Lokalisierung, Bahn- und Griffplanung werden durch das Paket gestartet.
