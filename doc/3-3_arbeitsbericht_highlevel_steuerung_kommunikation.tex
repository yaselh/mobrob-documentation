Zur Koordination der verschiedenen Softwarekomponenten der Gruppe wird ein Managementknoten benötigt, der gleichzeitig die Kommunikation mit anderen Gruppen leitet. Schnittstellen zwischen den Gruppen wurden in mehreren gemeinsamen Treffen einstimmig beschlossen und selbstständig vom jeweiligen Team implementiert. Für eine vereinfachte Konversation erhielt die vorgestellte Komponente den Namen \glqq Cup Acceptor\grqq , da die Tasse in Empfang genommen wird. Die Entwicklung findet zuerst unter dem Namen \glqq Highlevelcontrol\grqq statt, jedoch wird das Paket in \glqq cup\textunderscore acceptor\textunderscore manager\grqq umbenannt. Lokalisierung, Bahn- und Griffplanung werden durch das Paket gestartet. Zu Beginn stand eine Implementierung unter Verwendung von Services in Frage, jedoch war der Mehrwert von Actions schnell ersichtlich. Daher wurde bei ersten Tests die actionlib verwendet, um richtiges Verhalten auf den Schnittstellen der Komponente testen zu können.
