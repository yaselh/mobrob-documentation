Die erzielten Trajektorien waren für unseren Anwendungszweck annähern optimal. Zwar haben wir zwei Zwischenpunkte über den Zielen hinzugefügt, zwischen den einzelnen vorgegebenen Wegpunkte wurden allerdings die direkten Pfade gefunden und auch der Kaffee wurde dabei nicht verschüttet.
\newline
Obwohl unsere Bahnplanung vollständig zur Laufzeit stattfand, wurde die Trajektorien ohne merkbare Verzögerung berechnet und ausgeführt. Die Bahnplanung eines einzelnen Abschnittes hat dabei nie länger als 500ms für die Berechnung benötigt. Längere Verzögerungen in den Videoaufnahmen lassen sich mit der zusätzlichen Verzögerung von 500ms erklären, welche manuell eingefügt wurde, damit MoveIt genügend Zeit hat, um den momentanen Zustand des Roboterarms zu aktualisieren.
\newline \\
Bei der bisher verwendeten \textit{compute\_cartesian\_path}-Methodik gibt es jedoch einen entscheidenden Nachteil. Die Bahnplanung umfährt den Fuß des UR5 nicht eigenständig, sondern fährt gegebenenfalls den EEF über die Arm-Basis und beendet dann die Durchführung, da für die restliche Strecke keine Trajektorie mehr berechnet werden konnte. Bisher mussten wir also in manchen Situationen noch vordefinierte Umgehungspunkte hinzufügen. \textit{compute\_cartesian\_path} scheint nämlich den Arbeitsraum des UR5, welchen man ebenfalls für MoveIt anpassen kann, nicht bei der Planung zu berücksichtigen. Den Bereich über dem Roboterfuß als unzulässig zu deklarieren lieferte daher nicht den gewünschten Effekt.
\newline \\
Mit mehr Zeit würden wir als nächstes versuchen, die Bahnplanung gänzlich ohne Zwischenpunkte durchzuführen. Um dabei trotzdem die Tasse aufrecht zu halten, müsste man Pfad-Constraints hinzufügen und die Bahnplanung nur über die \textit{set\_pose\_target}-Methode durchführen.
\newline
Die so berechneten Trajektorien müssten, wie wir feststellen konnten, weiter optimiert werden, um zumindest annähernd die direkteste Bahn abzufahren. Mit der bisherigen Implementierung der \textit{set\_pose\_target}-Variante sind die Trajektorien nämlich oft noch zu ausschweifend. Da bei der Optimierung jedoch die Kinematik des Armes berücksichtigt werden muss, um die Machbarkeit zu gewährleisten und Kollisionen zu vermeiden, ist dies kein trivialer Nachbearbeitungsschritt.
\newline \\
Im bisherigen Stand der Bahnplanung ist außerdem keine Kollisionserkennung mit zusätzlichen Objekten der Umgebung implementiert. Diese Erweiterung wäre jedoch recht einfach, da das Planning Interface von MoveIt den Zustand der Szene überwacht. Könnte die Lokalisierung also für jedes Objekt auf dem Tisch und um den Tisch eine annähernd exakte Bounding Box bereitstellen, so müsste man diese nur über eine Methode zur Szene hinzufügen und schon würden sie bei der Bahnplanung mitberücksichtigt werden. Des Weiteren kann man genauso Bounding Boxen an den Roboterarm anbringen. Man könnte also auch eine Kollision der gegriffenen Tasse mit der Umgebung vermeiden.