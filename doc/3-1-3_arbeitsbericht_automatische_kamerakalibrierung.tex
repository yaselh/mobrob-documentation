So wie auch die Tassenerkennung wurde auch die Kalibrierung der Kamera ohne die Verwendung von fertig verfügbaren ROS Paketen realisiert. Es wurden zwei Ansätze realisiert wobei sich der als zweites implementierte Ansatz als deutlich robuster und genauer erwiesen hat und deshalb in das Gesamtsystem aufgenommen wurde.
Der zunächst implementierte Algorithmus basiert auf der Segmentierung der Tischplatte aus der Punktwolke. So wird als erster Schritt die Tischebene mittels des RANSAC Algorithmus aus der Punktwolke segmentiert. Der berechnete Normalenvektor der Eben wird anschließend dazu verwendet um die Tischplatte so zu rotieren, dass sie parallel zur x-y-Ebene ist. Daraufhin wird die minimale Bounding Box der Tisch Punktwolke im 2 Dimensionalen Raum berechnet und als Tischfläche angenommen. Es stellte sich beim testen des Algorithmus heraus, dass die Methode zwar schnell ist, jedoch zu ungenaue Ergebnisse liefert. Die Ungenauigkeiten waren wohl vor allem darauf zurückzuführen, dass die Punktwolke an den Ecken der Tischplatte einige Löcher aufwies und die Bounding Box so nicht den eigentlichen Umriss des Tisches wiedergab. Neben den Ungenauigkeiten bei der Tischdetektion ergab sich ein weiterer Nachteil aus den zur Ausführung des RANSAC Algorithmus verwendeten Python Bindings für PCL. So wurden diese, obwohl von der PCL Webseite verlinkt, augenscheinlich nicht mehr sehr aktiv gewartet und waren doch sehr unvollständig und teilweise fehlerhaft. Dies weckte zusätzlich den Wunsch die entsprechende Bibliothek aus dem Projekt zu entfernen.
So wurde schlussendlich ein zweite Ansatz implementiert. Dieser basiert auf der Detektion der Tischkanten aus den Bilddaten sowie einer anschließenden Segmentierung der die Tischkanten repräsentierenden Geraden in der Punktwolke. Beim testen des Ansatzes erwies sich dieser als deutlich genauer als die vorher implementierte Methode und auch die RANSAC Implementierung konnte mit geringem Mehraufwand, der durch die Implementierung einer geeigneten BaseEstimator Klasse entstand, durch die Implementierung von scikit-learn ersetzt werden. Die Details des schlussendlich implementierten Algorithmus sind in \ref{4-1-2_gesamtsystem_bilderkennung_automatische_Kamerakalibrierung} beschrieben.